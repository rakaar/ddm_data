 
++++++++++++++++++++++++++++++++++++++++++++++++++++++++++++++++++++++++++++++++++++++++++++++++++++++++++++++++++++
# Important equations and functions

def rho_A_t_fn(t, V_A, theta_A):
    """
    For Proactive process,prob density of t given V_A, theta_A
    """
    if t <= 0:
        return 0
    return (theta_A*1/np.sqrt(2*np.pi*(t)**3))*np.exp(-0.5 * (V_A**2) * (((t) - (theta_A/V_A))**2)/(t))


def cum_A_t_fn(t, V_A, theta_A):
        """
        For proactive, calculate cummulative distrn of a time t given V_A, theta_A
        """
        if t <= 0:
        return 0

        term1 = Phi(V_A * ((t) - (theta_A/V_A)) / np.sqrt(t))
        term2 = np.exp(2 * V_A * theta_A) * Phi(-V_A * ((t) + (theta_A / V_A)) / np.sqrt(t))
        
        return term1 + term2

def P_t_x(x, t, v, a):
    """
    Prob that DV = x at time t given v, a 
    """
    if t <= 0:
        return 0
    return (1/np.sqrt(2 * (np.pi) * t)) * \
        ( np.exp(-((x - v*t)**2)/(2*t)) - np.exp( 2*v*a - ((x - 2*a - v*t)**2)/(2*t) ) )

+++++++++++++++++++++++++++++++++++++++++++++++++++++++++++
# Important equations and functions

P(\frac{x, t}{v,theta})  =  (1/np.sqrt(2 * (pi) * t)) * \
        ( exp(-((x - v*t)**2)/(2*t)) - np.exp( 2*v*theta - ((x - 2*theta - v*t)**2)/(2*t) ) )


        

# Likelihood for Proactive drift jump at event of LED onset


## Terms used: 
\rho_A(\frac{t}{V, theta}) : Prob that proactive process hits the bound at time t given V and theta

P(x,t) = Prob that proactive process is at "x" at time "t"

CDF_A(\frac{t}{V, theta}) = \int_{0}^{t} \rho_A(\frac{t}{V, theta}) dt

## likelihood 
When LED onset occurs at $ t_{LED} $, the decision variable could be anywhere between $-\inf$ and \theta. 
Lets call the Decision variable at time $ t_{LED} $ as x, which could be any value below \theta.

From x, it has to now accumulate evidence with new drift V', and hit the bound which is at distance of \theta - x
If you are considering the reaction time as "t", then the new drift V' process has to hit \theta - x in time t - t_{LED}

So, Prob of hitting the bound at t, which is greater than t_LED
 = \int_{-\inf}^{\theta} P(x, t_{LED}) * \rho_A(\frac{t - t_{LED}}{V', \theta - x}) dx

 In short, likelihood can be written as:

 PDF_{\text{V change}}
 
t < t_LED: \rho_A(\frac{t}{V, \theta})
t > t_LED: \int_{-\inf}^{\theta} P(x, t_{LED}) * \rho_A(\frac{t - t_{LED}}{V', \theta - x}) dx

Another special case is when t_LED is zero, it means from the beginning itself, the process has started with new drift V'
t_LED == 0 \rho_A(\frac{t}{V', \theta})


# Fitting Proactive drift jump at event of LED onset on LED ON data


## Changes in model
- Upon LED onset, the proactive process drift jumps from V to V_', the new drift rate is higher and hence causes the 
proactive process to hit the bound earlier causing higher abort rate when compared to control trials
- Left Trunc and Right Censoring: Like in previous case, we want to left truncate the distribution at t_trunc, because we don't have a model for 
bi-modal distribution of aborts. We want to censor the RTs after t_stim, because we want to match the proportion of aborts.


## Left trunc and right censoring likelihood
t < T_trunc: 0
t < t_stim and t > T_trunc: PDF_{\text{V change}} / truncation_factor
t > t_stim: Censoring => (1 - CDF_{\text{V change}}) / truncation_factor

t_stim < T_trunc and t > t_stim: likelihood = 1 (because after t_stim, the area of all RTs is 1)
t_stim < T_trunc and t < T_trunc: likelihood = 0 (truncation)


























++++++++++++++++++++++++++++++++++++++++++++++++++++++++++++++++++++++++++++++++++++++++++++++++++++++++++++++++++++++++++++++++++++++++++++++++
# Left truncating and right censoring to fit Abort RTs

The stimulus onset time is varying and has an exponential distribution. We want to find paramters of Proactive process 
- V_A, theta_A and t_A_delay to fit the aborts. There are two issues:
1. The abort RT distribution is bi-modal, there is a peak before 0.3 s. Since we don't have a modal for this bi-modal type distribution
we are truncating the reaction times that are less than 0.3
2. We also want to match the proportion of aborts(number of aborts/number of trials), not just the shape of the distribution. 
For this we are censoring the likelihood of reaction times that are after the stimulus onset time. That means, all the reaction times
after a certain  $t_stim$ have same probability.

% ----- -figure of truncation and censoring ------ 

To address the above 2 issues, we are fitting on RTs such that the reaction times are left truncated before a certain time
 and right censored after $t_stim$


## calculating likelihood when you have data

For a trial
- If RT is less than T_trunc, then the likelihood is 0

- if a RT is before t_stim, it is an abort, we calculate the likelihood using the PDF of hitting time of Proactive process(call $\rho_A$).
But since we are truncating RT less than T_trunc, we need to adjust PDF by scaling it up. 



% ----- figure of truncating increasing likelihood  ------ 


Area under the new truncated PDF has to be 1. So, we scale up the PDF by dividing with 1 - CDF(T_trunc)

likelihood of an abort accounted for truncation = PDF(t) / (1 - CDF(T_trunc))


- If RT is greater than t_stim, we assign the RT the probability of all samples that come after $t_stim$

Likelihood of a valid trial RT accounting for censorship = 1 - CDF(t_stim)



# But once you have the parameters obtained from VBMC, we need to test if the histogram of the RTs fits well with the theoretical PDF

There are 2 ways to calculate the theoretical PDF of aborts

1. you know the distribtuon of stimulus onset:

The stimulus onset time distribution (PDF_stim)= 1/ tau * exp( - ( t - t_0 ) / tau )
Prob that stim survives till time "t" is 1 - prob that stim occured before "t"
 =  1 - \int_{t_0}^{t} PDF_stim(t) dt = exp( - ( t - t_0 ) / tau )


 Prob of an abort at time t = Prob that stim survives till "t" & Prob that Proactive hits the bound
                             = exp(- (t - t_0) / tau ) * PDF_hitting_single_bound(t)


Note that you have account for truncation by dividing 1 - CDF of hitting abort at T_trunc

2. you don't know/want to assume the distribution of stimulus onset:

- Collect all the t_stim from the data
- for each t_stim, calculate the PDF of hitting abort at time "t"(NOT t_stim) multiplied by if stim survived till t_stim or not (t > t_stim or t < t_stim)
- Finally take average of all PDFs calculated for each t_stim
PDF of abort at time (t) = 1/ N \Sigma_{i=1}^{N} PDF_hitting_single_bound(t) * Stim survived till t or not 


# Note that once you have calculate the theoretical PDF, you can compare the shape of the histogram, 
but to check if proportions match

- normalize the histogram of aborts with total number of trials (multiply the histogram with num of aborts/ num of trials)
- normalize the theoretical PDF with fraction of aborts calculated theoretically. 

The fraction of aborts is calculated theoretically by:
- For each t_stim, calculate the non-censored probability(account for truncation) = 1 - CDF_of_hitting_single_bound(t_stim)
- Take average of all the non-censored probabilities calculated for each t_stim
- If done correctly, the above average should match the fraction of aborts in the data

Fraction of aborts = 1/N \Sigma_{i=1}^{N} 1 - CDF_of_hitting_single_bound(t_stim_i)   




\section{Left and right trunc VS. Left trunc and right censoring}
Instead of censoring RTs after $t_{stim}$, if we truncate RTs after $t_{stim}$, 
the credible intervals of the parameters(16th and 84th percentiles of VP samples) are wider because we are reducing the amount of 
information to the VBMC fit.

% figures of left trunc, right trunc image LTRT.png and left trunc and right censoring LTRC.png side by side